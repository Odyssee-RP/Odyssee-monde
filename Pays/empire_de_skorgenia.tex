\documentclass[french, a4paper, 12pt]{article}

\newcommand{\docname}{Empire de Skorgenia}

% Layout
\usepackage{fancyhdr, fancyvrb}
\usepackage{lastpage}
\usepackage[left=2.0cm, right=2.0cm, top = 2.0cm, bottom = 2.0cm]{geometry}
\usepackage{multicol}
\usepackage{graphicx}
\usepackage{hyperref}
\hypersetup{hyperindex=true, colorlinks, linkcolor=blue, urlcolor=blue, citecolor=blue, breaklinks=true}
\everymath{\displaystyle}


% Language
\usepackage[utf8]{inputenc}
\usepackage[T1]{fontenc}
\usepackage{babel}


% Pagestyle
\pagestyle{fancy}

\renewcommand{\headrulewidth}{0.4pt}
\renewcommand{\footrulewidth}{0.4pt}

\title{\sc \docname}
\date{}

\lhead{}
\rhead{\docname}
\cfoot{\thepage\ $|$ \pageref{LastPage}}
\rfoot{Odyssée}

\begin{document} \maketitle \vspace{3pt} \hrule \vspace{3pt}

\section{Chiffres}

\begin{verbatim}
Population ...........: 6.6 millions
Superficie ...........: 911 000 kilomètres carrés
Régime politique .....: Collège (un représentant par guilde)
Capitale .............: Fludir
Adjectif .............: skorgais, skorgaise
Culture ..............: Norse, Yotunn
Religions pratiquées .:
\end{verbatim}

\section{Culture}

\subsection{Généralités}

La culture Norse domine la partie sud de l'Empire de Skorgenia, mais le Yotunn étend son influence au nord du pays. Cette division est sensible au niveau de la langue, où le yotunn skorgais reste compris par une large partie de la population du nord, alors que seul le norse standard est parlé au sud. Toutefois, le système éducatif supporté par les guildes tend à faire diminuer l'usage du yotunn skorgais au profit du prestigieux yotunn classique comme langue académique et religieuse et du norse standard come langue de communication et langue administrative.

\section{Société}

\subsection{Généralités}

Le peuple skorgais vit dans un climat froid. La société est organisée, selon les principes de la culture Norse adoptés par l'État, c'est à dire avec des guildes. Cette organisation sociétale permet une séparation claire des différents corps de métier ce qui garantit une grande efficacité.

\subsection{Système de guildes}

Les liens entre les différentes guildes sont ténus et, traditionnellement, les enfants restent dans la même guilde que leurs parents, eux-même généralement issus de la même guilde. Les mariages inter-guildes ne sont pas formellements interdits, mais plutôt mal vus. Les points positifs à cette organisation sévère ne manquent pas : l'État a imposé un niveau minimal d'éducation, ainsi tous les enfants savent lire, écrire et compter. Vers l'âge de 10 ans ils sont intégrés dans la guilde de leur choix, mais c'est à l'enfant de trouver un maître pour terminer sa formation. Quelques enfants plus curieux ou ouverts vont chercher des maîtres, mais la plupart restent avec leurs parents.

Autre point positif : le faible taux de criminalité, certaines guildes sont chargées de la surveillance et sont efficaces. De manière générale, les bourgs ont une guilde majoritaire qui fait office de corps municipal, gérant la ville, son approvisionnement et sa sécurité.

\subsection{La guilde des Marchands}

Il s'agit de la guilde la plus puissante et la plus riche. L'absence de grande ville marchande fait toutefois qu'elle ne contrôle aucune cité en propre.

\section{Politique}

\subsection{Généralités}

Le système est rigoriste, autoritaire et centralisé, mais le bon niveau de vie ainsi qu'une importante fierté nationale font que la population en est contente. Les conséquences en sont un chauvinisme presque belliqueux mais aussi un taux de criminalité remarquablement bas. Le peuple est en harmonie avec son gouvernement.

\subsection{Système politique}

L'Empire est dirigé par un collège au sein duquel chaque guilde est représentée par un seul est unique membre (dont le mode de désignation diffère selon la guilde). La présidence du Collège est depuis longtemps assurée par la guilde des Marchands, non pas que ce poste lui revienne de droit, mais parce qu'il sagit de la guilde la plus riche, la plus influente et donc la plus puissante.

Cette dernière aimerais asseoir son pouvoir sur l'Empire entier et sans avoir à passer par le vote collégial pour faire appliquer des décisions. Des tensions internes au gouvernement naissent ainsi. Mais celles-ci restent éphémères et trop ténues pour engendrer une véritable crise politique. Mais si la guilde des Marchands venait à faire un faux pas dans la politique interne, les autres guildes ne lui pardonneraient pas, et une lutte inter-guilde pour gouverner le collège pourrait avoir lieu.

\section{Économie}

\subsection{Généralités et fonctionnement de base}

L'économie skorgaise repose sur l'exploitation de la forêt, le travail de fer et de la pierre. Les matière premières sont importées par la guilde des Marchands, elles sont ensuite acheminées vers la guilde qui va la travailler. Par exemple la guilde des Menuisiers, des Orfèvres, des Forgerons ou encore des Tailleurs de pierre.

Les pièces terminées sont ensuite reprises par la guilde des Marchands qui les redistribuent dans le monde entier à partir des quatres ports principaux de l'Empire de Skorgenia : Nozigkezda, Dizen, Durilkun et Tozig.

\subsection{Matières premières}

Si le pays importe une bonne partie de sa matière première, l'empire skorgais exploite néanmoins une mine de fer à proximité de la ville de Toldonzir. Le bois est en partie importé depuis le Grand Duché de Boisia.

\subsection{L'assassinat : un art dirigée par le Collège}

Une autre part de l'économie est le service de mercenaires. Dirigée directement par l'État, la guilde des Assassins (5ème régiment) est à la fois une police interne, un service de renseignement, un corps de l'armée skorgaise et un groupe de mercenaire dont le Collège vend les services. À l'origine la guilde des Assassins est un détachement du second régiment, un corps d'élite du nom de Loyanesia directement attaché au gouvernement, comprendre à la botte de la guilde des Marchands.

\section{Diplomatie}

\subsection{Généralités}

La culture Norse est assez radicale, et le peuple, sans être belliqueux, est fier de son organisation, un chauvinisme qui créé des tensions entre les pays frontaliers. Récemment les tensions avec l'Empire d'Otlin le long de la frontière nord se sont aggravées.

\subsection{Historique des relations internationales}

Le dernier fait majeur de ces dernières années est sans doute la guerre de territoire qui a opposé l'Empire de Skorgenia et la Principauté d'Haurin qui s'est soldée par une victoire skorgaise.

Suite à cette guerre, la géopolitique internationale a été redessinée : l'Empire de Skorgenia a vu ses relations avec la Principauté d'Haurin et le Royaume de D'elkisad se dégrader fortement. À l'inverse ses relations avec la Principauté d'Uther, le Grand Duché du Ched et l'Empire d'Omar se sont considérablement améliorées.

Certains pays restent plutôt opposés l'Empire de Skorgenia, par intérêt politique ou par alliance, c'est le cas du Grand Duché de Willow, et du Royaume de Lidiel.

À l'inverse, d'autres pays soutiennent l'empire skorgais par intérêt financier ou militaire.

Quelques pays de l'est, trop éloignés pour être concernés, même par le jeu des alliances, conservent une neutralité qui pourrait faire pencher la balance de manière définitive pour l'avenir de l'Empire de Skorgenia si ils venaient à choisir un camps.

\subsection{Invasion de l'Empire d'Otlin}

\subsubsection{Contexte}
L'Empire a récemment entreprit d'envahir le sud de d'Otlin pour étendre son influence. Le combat se déroulant majoritairement dans des petits villages montagnars, Otlin a été long à réagir et n'a envoyé que peu d'hommes, mobilisant un unique régiment de 1600 hommes alors en garnison à Gigezruc, situé à 300 kilomètres au nord de la zone de conflits.

Pendant ce temps le Royaume a massé près de 2000 hommes réparti en deux régiments distincts et chacun attaquant un endroit différent. La zone occupée ne cesse de s'étendre (aux dernières nouvelles la zone sous contrôle avoisinnait les 38~000 kiloèmtres carrés) et le plus proche régiment Otlinois se trouve à plus 1500 kilomètres.

\subsubsection{Bataille de la rivière de Kinzarkud}
Le détachement skorkais fort d'un millier d'hommes ne pensait pas se faire attaquer par un régiment plus grand. L'attaque de l'Empire d'Otlin a été rapide et efficace. L'armée skorgaise a essuyé une cuisante défaite, mais le régiment otlinois a été réduit aussi. Néanmoins, cette bataille restera une humiliante défaite pour l'armée skorgaise.

\end{document}