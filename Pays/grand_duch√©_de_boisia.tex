\documentclass[french, a4paper, 12pt]{article}
\newcommand{\docname}{Grand Duché de Boisia}
% Layout
\usepackage{fancyhdr, fancyvrb}
\usepackage{lastpage}
\usepackage[left=2.0cm, right=2.0cm, top = 2.0cm, bottom = 2.0cm]{geometry}
\usepackage{multicol}
\usepackage{graphicx}
\usepackage{hyperref}
\hypersetup{hyperindex=true, colorlinks, linkcolor=blue, urlcolor=blue, citecolor=blue, breaklinks=true}
\everymath{\displaystyle}


% Language
\usepackage[utf8]{inputenc}
\usepackage[T1]{fontenc}
\usepackage{babel}


% Pagestyle
\pagestyle{fancy}

\renewcommand{\headrulewidth}{0.4pt}
\renewcommand{\footrulewidth}{0.4pt}

\title{\sc \docname}
\date{}

\lhead{}
\rhead{\docname}
\cfoot{\thepage\ $|$ \pageref{LastPage}}
\rfoot{Odyssée}

\begin{document} \maketitle \vspace{3pt} \hrule \vspace{3pt}

\section{Chiffres}

\begin{verbatim}
Population ...........: 899 000
Superficie ...........: 183 000 kilomètres carrés
Régime politique .....: Anarchie post-révolutionnaire par le peuple
Capitale .............: Vinou
Adjectif .............: boisais, boisaise
Culture ..............: Luarian
Religions pratiquées .: Les Témoins de Griverdeneres
\end{verbatim}

\section{Culture}

\subsection{Généralités}

La culture Luarienne repose sur la culture Trow. Proche du peuple et assez ouverte, elle a largement contribuée aux événements internes du Grand Duché. L'ouverture de cette culture a permit une réelle coallition du peuple, les différentes ehtnies se sont soudées entre elles.

\section{Société}

\subsection{Généralités}

Le peuple est par nature assez libre et impulsif. Cependant, si le peuple est assez sanguin, c'est aussi parce que le gouvernement était faible et mou. Énervé par sa lenteur, le peuple a pris les armes et a renversé son gouvernement.

\subsection{Problèmes actuels}

La révolution s'est faite piloter par le Royaume de Seresia qui profite du manque d'organisation pour faire chuter l'économie et rendre le pays de plus en plus faible dans le but de l'annexer pour utiliser les ressources.

\section{Politique}

\subsection{Ancien gouvernement}

Accusé de ne pas bien gérer le pays et ses ressources, le peuple, composé de différentes ethnies, s'est révolté et à renversé son gouvernement avec l'aide logistique et matérielle du Royaume de Seresia.

En réalité, le gouvernement boisais était coincé entre ses puissants voisins et son économie en berne, la situation était déjà dure à gérer, avec la montée de la colère du peuple, l'économie a encore chuté et la situation n'a cessé de s'aggraver, accentuant encore le ressentiment de la population vis à vis de son Duc.

\subsection{Révolte et nouveau gouvernement}

En 285, les tensions ont éclatés : les grandes villes ont été saisies rapidement par le peuple armé, les dirigeants et tout leurs proches ont été pourchassés et assassinés. Le peuple s'est alors retrouvé devant la situation du Grand Duché : un pays sans économie, sans gouvernement, entouré de faux amis et d'alliances dangereuses.

Le peuple a alors dû prendre des décisions dans l'urgence pour sauver l'indépendance de Boisia tout en relevant l'économie. La volonté initialle était de créer un Conseil, mais actuellement, c'est une sorte d'anarchie qui règne.

Le pays est co-géré par le Royaume de Seresia qui compte réduire la population de Boisia en esclavage, non pas au nom d'un quelconque désir de domination, mais pour assouvir un besoin de richesse, un appât du gain et de l'argent, et ce à n'importe quel prix. Dans son désir de s'octroyer le territoire boisais, Seresia s'arrange pour maintenir le pays dans une anarchie crasse pour pousser à une seconde révolte qui servira de pretexte à une prise totale et définitive du pays.

\subsection{Tensions internes}

L'influence grandissante du Royaume de Seresia dans les décisions internes de Boisia énerve la population qui aimerait plus d'indépendance, mais leur position ne leur permet pas de rompre cette alliance.

\section{Économie}

\subsection{Bases de l'économie}

Par sa position et sa forme en presqu'île, le Grand Duché de Boisia n'est constitué que de forêts tempérées et humides avec des précipitations annuelles de l'ordre de 4 000 millimètres. Le pays a des ressources en bois de différentes essences et son exploitation permet à l'économie boisaise de survivre, mais n'est pas suffisante pour un réel épanouissement du pays.

\subsection{Limites de l'économie}

Le Grand Duché n'a que très peu d'universités et pour ainsi dire aucun centre de recherche, le bois est coupé, mais n'est donc pas travaillé sur place. Une faible agriculture vivière permet aux habitants d'avoir de quoi vivre, l'importation de nourriture étant impossible au vu des délais de conservation trop court.

\subsection{Problèmes posés}

L'économie boisaise chute sous les yeux attentifs des pays voisins, notamment l'Empire de Seresia, qui attend la déchéance totale de l'économie vacillante du petit pays pour en prendre définitivement le contrôle total.

\section{Diplomatie}

\subsection{Généralités}

Le gouvernement boisais s'est fait renverser en 285 et le pays n'a pas encore eu le temps de nouer de réelles alliances.

Néanmoins, certains pays ont senti que le changement de régime pouvait leur apporter sur le plan économique, aussi certains pays ont proposé leur aide au peuple boisais, c'est le cas de l'Empire d'Omar, de la principauté stanlandaise, du Royaume D'elkisad et de l'Heptarchie d'Orle. Aucun pays n'a directement pris la défense de l'ex-gouvernement, mais les anciennes relations qui soutenaient l'État sont plus tendues, c'est le cas pour le Grand Duché du T'lind, de la Théocratie de Torkia et de la Fédération du Bolagref.

\subsection{Royaume de Seresia}

Lorsque le peuple boisais a commencé à vouloir renverser son gouvernement, le Royaume de Seresia a proposé d'équiper les révolutionnaires et de les aider. Peu après la révolution, Seresia s'est arrangée pour placer des Hommes a elle aux hautes fonctions boisaises. Profitant de la faiblesse du Grand Duché de Boisia, Le Royaume profite du peuple et de ses ressources pour s'enrichir.

Ce rapport de force, imposé par l'armée seresienne, fait du Royaume de Seresia le suzerain de Boisia, qui, impuissant, ne peut rien faire pour l'instant.


\end{document}