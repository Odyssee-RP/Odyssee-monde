\documentclass[french, a4paper, 12pt]{article}

\newcommand{\docname}{Histoire du monde}

% Layout
\usepackage{fancyhdr, fancyvrb}
\usepackage{lastpage}
\usepackage[left=2.0cm, right=2.0cm, top = 2.0cm, bottom = 2.0cm]{geometry}
\usepackage{multicol}
\usepackage{graphicx}
\usepackage{hyperref}
\hypersetup{hyperindex=true, colorlinks, linkcolor=blue, urlcolor=blue, citecolor=blue, breaklinks=true}
\everymath{\displaystyle}


% Language
\usepackage[utf8]{inputenc}
\usepackage[T1]{fontenc}
\usepackage{babel}


% Pagestyle
\pagestyle{fancy}

\renewcommand{\headrulewidth}{0.4pt}
\renewcommand{\footrulewidth}{0.4pt}

\title{\sc \docname}
\date{}

\lhead{}
\rhead{\docname}
\cfoot{\thepage\ $|$ \pageref{LastPage}}
\rfoot{Odyssée}

\begin{document} \maketitle \vspace{3pt} \hrule \vspace{3pt}
	
\tableofcontents

\section{Historique des cultes et religions}

	On distingue quatres types de religions~:
	\begin{itemize}
		\item les religions organisées qui suivent une hierarchie interne stricte et établies.
	 	\item les religions populaires n'ont pas d'ordre pré-établi et avec une liberté plus grande laissée aux praticants. 
	 	\item les cultes obéissent à des règles de hierarchie internes. Là où les autres religions se focalisent sur des dieux, les cultes sont plutôt tournés vers des personnes dont l'existence est avérée. Certains cultes se politisent en prenant parti pour une cause et deviennent des sectes.
	 	\item les hérésies sont des religions anarchiques dérivées d'une autre religion.
	 \end{itemize}

	\subsection{Panthéon de Rangorhe}

		La première religion a être apparue est le Panthéon de Rangorhe, une religion polythéïste dont la divinité suprême est Elkegke, surtout présente à l'ouest du monde, dans les environs de l'Empire d'Otlin. Malgrè son ancienneté, le Panthéon de Rangorhe compte encore aujourd'hui plus de 9 millions de fidèles, soit 21\% de la population mondiale, réparti sur un territoire de près de 1.5 millions de kilomètres carrés (22\% de la surface du monde). Cette religion primordiale s'est très vite subdivisée en deux courants de pensée antagonistes~: les premiers ont évolués vers une religion monothéïste et plus violente, tandis que les seconds ont priviligiés une pensée polytheïste.
		Les officiants du Panthéon sont des druides. Ceux de l'Empire d'Otlin sont désignés et officiellement ordonnés par l'Église impériale, à la tête de laquelle se trouve l'Empereur lui-même. Ce sont parfois des druides de l'Église impériale qui officient dans les villages des Empires d'Omar et de Skorgenia, mais le très traditionnel Empire d'Otlin n'a jamais vu en eux un outil de propagande culturelle. De voix s'élèvent toutefois à Elkegkan pour faire de ces druides un outil politique majeur dans le conflit larvé qui oppose Otlin à Skorgenia.
		Les druides de Skorgenia sont toutefois souvent des locaux, formés au Sanctuaire skorgais de Kinyle.

	\subsection{Le Raknid}

		Le courant monotheïste s'est très vite séparé du Panthéon avec l'arrivée de Raknid sorte de prophète envoyé parmi les Hommes par Ziavashu pour défendre son droit au trône du Panthéon. En effet, Zhiavashu le Sévère, dieu mineur du Panthéon de Rangorhe, a été, selon la légende, spolié de ses droits au trône du Panthéon. Raknid aurait donc répandu sa parole pour plaider sa cause. Aujourd'hui disparue, cette religion a donné lieu à trois courants différents, mais tous monotheïstes.

		\subsubsection{Le Mythe eldarien}

		Le Mythe d'Eldar, est une religion populaire, issue du Raknid, qui reste dans la lignée des rites shamaniques du Raknid. Présente sur un très petit terrioire au nord, ce courant a évolué vers le culte ancêstral Norse, une religion du peuple dans laquelle on distingue le culte sombre de Selveloyrien, une guilde d'assassins, et la Secte de Gartadie. Le Mythe d'Eldar est également considéré comme le père de la Secte des Hlauggaustiens, aujourd'hui éteinte.

		\subsubsection{Les Témoins de Griverdeneres et de Hluirerarith}

		Les Témoins de Griverdeneres et de Hluirerarith sont les deux derniers courants de pensées issus du Raknid. Avec plus de 15 millions de croyants répartis sur 2.3 millions de kilomètres carrés, Les Témoins de Griverdeneres est la religion la plus répandue et étendue. Cette religion organisée suit une hierarchie qui lui est propre en prônant la foi en Griverdeneres, le dieu unique.

	\subsection{La voie polythéïste}

	Un second courant polytheïste, plus proche du Panthéon de Rangorhe a emmergé environ au même moment que le Raknid. Ce courant s'est assez vite divisé en deux, avec d'un côté la Voie de Solfond et de l'autre, la Parole de Tritaranlean. Si le Solfond présente peu d'intérêt, la Parole de Tritaranlean est une religion qui s'est mieux développée.

	Cette dernière a donné naissance à un panthéon mineur~: les Dieux du Luarian, dont les 133 milles membres sont répartis entre le Grand Duché de Boisia et le Royaume de D'elkisad. Mais aussi et surtout deux religions manichéennes la religion Trow (intimement lié à la culture du même nom) et le Jhaclonese. Cette dernière s'est subdivisé en deux courants mineurs et polytheïstes qui rompent avec le dualisme du Jhaclonese~: d'une part la Foi d'Enland avec seulement 5800 fidèles, c'est l'une des religions les plus rares, endémique à une petite enclave à la pointe nord-est de la Théocratie de Torkia. Et d'autre part la religion populaire Romian.

	\subsection{Diagramme récapitulatif}

	\begin{verbatim}
		Panthéon de Rangorhe ............. (polytheïste, organisée)
		| La Parole de Tritaranlean ...... (polythéïste, organisée)
		|   |-- Jhaclonese ............... (manichéenne, organisée)
		|   |   |-- La Foi d'Enland ...... (polythéïste, populaire)
		|   |   |-- Romian ............... (polytheïste, populaire)
		|   |-- Les Dieux du Luarian ..... (polytheïste, populaire)
		|   |-- Trow ..................... (manichéenne, organisée)
		|-- Raknid - disparue ............ (shamanes, populaire)
			|-- Le Mythe d'Eldar ............. (shamanes, populaire)
			|   |-- Culte Norse .............. (culte ancêstral, populaire)
			|       |-- Secte de Gatardie .... (culte, culte)
			|       |-- Selveloyrien ......... (culte sombre, culte)
			|-- Les Témoins de Griverdeneres . (monothéïste, organisée)
			|-- Les Témoins de Hluirerarith .. (monotheïste, organisée)
	\end{verbatim}

\section{Histoire culturelle}

	\subsection{La culture Yotunn}

		Historiquement, la première culture est celle du Yotunn, restée confinée dans l'Empire d'Otlin, elle touche 8.8 millions de personnes pour un territoire de 1.4 millions de kilomètres carrés. Au fil des ans la culture s'est peu à peu répandue vers le sud touchant ainsi l'Empire de Skorgenia, mais aussi, de manière plus minoritaire la Principauté de Haurin et le Lidiel. Cette culture s'est divisée en plusieurs courants plus ou moins importants.

		D'un point de vue linguistique, la culture Yotunn est très ancienne et repose en grande partie sur un corpus d'écrits rédigés en haut-yotunn, une langue aujourd'hui disparue mais dont sont dérivés les nombreux dialectes populaires parlés partout dans l'Empire d'Otlin, ainsi que le yotunn classique, langue académique, universitaire et administrative dans laquelle sont rédigés les actes officiels de l'Empire.
		Au nord de l'Empire de Skorgenia, où la culture Yotunn est également largement diffusée, les populations parlent encore souvent le yotunn skorgais, proche des idiomes otlinois méridionaux, bien que celui-ci tende à disparaitre au profit du bilinguisme yotunn classique métissé de norse standard, la première étant plus prestigieuse et la deuxième largement diffusée par l'excellent système éducatif skorgais et présentant l'intérêt d'être assez simple d'apprentissage.

		Nous allons passer rapidement sur les pensées mineures à savoir la culture Aj'Snaga, présente au nord de l'Empire Omarien sur une langue de terre d'un peu moins de 9 900 kilomètres carrés, mais aussi la culture Romian qui s'est sublimée en la culture Kentian.

		Ces trois cultures sont peu connues et difficiles d'accès, et par leurs règles qui suivent des principes strictes mais aussi par leur positions qui induit un climat plus froid.

	\subsection{La culture Eldar}

		Monument culturel majeur, édifice indétrônable au prestige inégalable, la culture Eldar regroupe les grandes puissances du nord avec en tête l'Empire d'Omar, Torkia, Stanland et Orle. Si deux des trois courants de pensées issue de l'Eldar sont restés au nord, le rayonnement culturel Eldar est tel qu'un écho de cette culture est né au sud-ouest de l'Empire de Skorgenia~: la culture Norse. Mais cette dernière a perverti la pensée initiale et bien qu'historiquement liées, les philosophies Eldar et Norse n'ont plus grand chose en commun. En effet, la culture Norse, très influencée par la religion éponyme est très orientée vers un système social organisé en guildes dont certaines peuvent être très violentes.

		La culture Norse a malheuresement trouvé des adeptes et des disciples dans deux pensées mineures~: le Durinn et le Drake, toute deux des religions dont les centres culturel sont situés sur des hauts plateaux, aux conditions de vie plus difficiles.

	\subsection{Le Trow}

		Le dernier courant de pensée dérivé du Yotunn est la culture Trow, partiellement issue de la religion du même nom, mais aussi de métissage d'autres religions. La culture Trow est considérée comme un idéal de liberté cosmopolite. Les religions et les cultures sont entremêlées et vivent plus ou moins en harmonie. Némanmoins il existe des conflits interne à cette culture. La principale lutte intestine concerne l'emplacement de la capitale culturelle~: Shaninarra, cette dernière est située à la frontière D'elkisado-wilowienne, et a fait l'objet d'une guerre. Malgré ces luttes, la culture Trow est la plus répandue et dans l'espace, avec un rayonnement sur un territoire 1.5 millions de kilomètres carrés, réparti sur 8 pays différents et de religions diverses, mais aussi en terme de nombre de personnes touchée avec 11.8 millions d'adeptes.

		La pensée Trow, ouverte dans son fond est plutôt permissive et a donnée naissance a des philosophies tournées vers les autres. Mais si la culture est attrayante et la philosophie prônée raisonnable, ce n'est pas un dogme étatique et les volontés des gouvernements divergent quand il s'agit d'asseoir sa puissance culturelle et son influence.

		\subsubsection{Courants de pensées dérivés}

		Le Trow s'est subdivisé en trois courant assez mineurs par leur taille et le nombre d'adeptes, on citera néanmoins la culture Rakhnid, Luarianienne, et Uruk. Ces trois courants sont restés localisés au sud et au sud-est sans toucher beaucoup de monde, leur rayonnement étant étouffé par la puissance du Trow à l'est et du Norse à l'ouest. La culture du Luarian a cependant réussi à conquérir Seresia.

		Les relations entre ces quatre cultures et les religions sont assez complexes. La religion Trow est à l'origine de la culture Trow qui a ensuite séduit des populations voisines d'autres confessions, lesquelles ont alors suivi, selon les cas, deux voies distinctes~: soit accepter la culture Trow en y apportant une part de la religion de départ, contribuant ainsi à l'élargissement et à l'enrichissement de la culture ; soit fonder une autre culture basée sur le Trow mais qui serait plus proche de la religion de départ.

	\subsection{Diagramme récapitulatif}

		\begin{verbatim}
			Yotunn
			|-- Aj'Snaga
			|-- Eldar
			|   |-- Enlandic
			|   |-- Norse
			|   |   |-- Drake
			|   |   |- Durinn
			|   |-- Nortumbic
			|-- Romian
			|   |-- Kentian
			|-- Trow
				|-- Luarian
				|-- Rakhnid
				|-- Uruk
		\end{verbatim}

\section{Histoire géopolitique}

	\subsection{Première guerre D'elkisado-willowienne}

		\subsubsection{Tensions de départ}

		Le peuple Trow est une ethnie de chasseurs dont la capitale culturelle se trouve entre le Royaume D'elkisad et le Grand Duché de Willow. Les Trows de D'elkisad et de Willow revendiquant chacun la "vraie" culture Trow.

		\subsubsection{Déroulement de la guerre}

		A cause de ces tension, en l'an 124, le Royaume de D'elkisad déclare la guerre au Grand Duché de Willow, son rival. Fort d'une armée de plus de 40 000 Hommes, D'elkisad progresse rapidement dans le Grand Duché. Lequel est laissé seul depuis que le Royaume de Seresia s'est retiré du pacte d'aide mutuelle. Mais si les alliés de D'elkisad ne répondent pas à l'appel au combat, cela n'empêche pas le Royaume d'écraser l'armée Willowienne.

		Voyant le Grand Duché au bord de la défaite, Seresia retourne sa veste et signe une alliance avec D'elkisad. Traversant alors la Principauté d'Uther, alliée à D'elkisad, l'armée Seresienne prend en tenaille les troupes déjà fragilisées du Grand Duché. Par le jeu des alliances, le Grand Duché de Boisia et l'Heptarchie Orlaise entrent en guerre au côté de D'elkisad finisant de réduire l'armée Willowienne. 

		\subsubsection{Résultats}

		À la fin de la guerre D'elkisado-willowienne, les troupes de Seresia et de ses vassaux se sont retirés, laissant le Royaume D'elkisad occuper seul la partie sud-ouest du pays. Depuis, l'armée Willowienne est devenue l'une des plus puissantes du monde, forte d'un désir de vengeance, elle accumule les victoires contre l'occupant D'elkisadais. Aujourd'hui le Grand Duché de Willow est le pays le plus militarisé, et a l'une des armées les plus active. Le territoire sous occupation D'elkisadaise est encore de 29 milles kilomètres carrés, une occupation qui concerne un peu plus de 200 000 habitants.

	\subsection{La Guerre Skorga-haurienne}

		\subsubsection{Tensions de départ}

		Le territoire Trow s'étend jusqu'à l'ouest de la Principauté de Haurin où la culture Norse apparaît. 
		Toutes les religions sont issues d'une pensée unique~: le Panthéon de Rangorhe, lequel à donné naissance à deux courants majeurs dont découlent les religions actuelles. L'un des courants a suivi une idéologie monothéiste et sectaire qui a mené à la Secte de Gartadie, l'autre est resté sur une pensée polythéïste comme les fidèles de la Parole de Tritaranlean.

		\subsubsection{Déroulement de la guerre}

		La Secte de Gartadie, située à l'extrémitée sud-est de l'Empire de Skorgenia, a déjà tenté quelques actions militaire contre les fidèles de Tristaranlean, mais sans jamais parvenir à installer une colonie durable. Ainsi, en 290, l'Empire de Skorgenia rassemble 8 515 soldats et attaque Haurin qui oppose alors près de 7 000 Hommes. La guerre fait rage et, si la Principauté d'Uther refuse d'entrer en guerre auprès de Haurin, le Royaume de D'elkisad répond favorablement à l'appel. Haurin et D'elkisad ont une culture Trow commune, alors que Skorgenia suit une culture Norse, il s'agit donc d'une occasion pour D'elkisad d'étendre son influence culturelle.

		Devant ce retournement de situation, l'Empire de Skorgenia tente de faire appel à Uther, mais la principauté refuse de nouveau d'entrer en guerre. Au bout d'un bon mois de préparatif, l'armée D'elkisadaise, enfin prête, traverse le Ched pour rejoindre les combats, commettant au passage des atrocités sur les populations locales. Le Ched, jusqu'ici neutre, répond alors favorablement à la demande d'aide de l'Empire de Skorgenia et prend à revers les troupes Haurinoises.

		\subsubsection{Résultats}

		Malgré la résistance opposée à l'Empire et ses alliés, la Principauté de Haurin perd définitivement la pointe sud-ouest de son territoire, une surface d'environ 71 000 kilomètres carrés. 


\end{document}